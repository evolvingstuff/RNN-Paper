\documentclass{article}

% ready for submission
\usepackage{nips_2018}

% to compile a preprint version, e.g., for submission to arXiv, add
% add the [preprint] option:
% \usepackage[preprint]{nips_2018}

% to compile a camera-ready version, add the [final] option, e.g.:
% \usepackage[final]{nips_2018}

% to avoid loading the natbib package, add option nonatbib:
% \usepackage[nonatbib]{nips_2018}

\usepackage[utf8]{inputenc} % allow utf-8 input
\usepackage[T1]{fontenc}    % use 8-bit T1 fonts
\usepackage{hyperref}       % hyperlinks
\usepackage{url}            % simple URL typesetting
\usepackage{booktabs}       % professional-quality tables
\usepackage{amsfonts}       % blackboard math symbols
\usepackage{nicefrac}       % compact symbols for 1/2, etc.
\usepackage{microtype}      % microtypography

\author{
  Lahore, Thomas\\
  \textt{tom.lahore@gmail.com}
  \and
  Waver, Morgan\\
  \textt{morganjweaver@gmail.com}
}

%
 \title{Diagnet: A New Recurrent Neural Network Architecture}
\author{
Tom Lahore
\and Morgan Weaver
}
% The \author macro works with any number of authors. There are two
% commands used to separate the names and addresses of multiple
% authors: \And and \AND.
%
% Using \And between authors leaves it to LaTeX to determine where to
% break the lines. Using \AND forces a line break at that point. So,
% if LaTeX puts 3 of 4 authors names on the first line, and the last
% on the second line, try using \AND instead of \And before the third
% author name.

%\author{
%  Tom Lahore\thanks{Use footnote for providing further
%    information about author (webpage, alternative
%    address)---\emph{not} for acknowledging funding agencies.} \\
%  Department of Computer Science\\
%  Cranberry-Lemon University\\
%  Pittsburgh, PA 15213 \\
%  \texttt{hippo@cs.cranberry-lemon.edu} \\
%  \And
%  Morgan Weaver \\
%  }
  %% Affiliation \\
  %% Address \\
  %% \texttt{email} \\
  %% \AND
  %% Coauthor \\
  %% Affiliation \\
  %% Address \\
  %% \texttt{email} \\
  %% \And
  %% Coauthor \\
  %% Affiliation \\
  %% Address \\
  %% \texttt{email} \\
  %% \And
  %% Coauthor \\
  %% Affiliation \\
  %% Address \\
  %% \texttt{email} \\


\begin{document}
\maketitle
\nolinenumbers
\begin{abstract}

  Recurrent Neural Networks have long been plagued by the vanishing and exploding gradient problems and limited performance over increasing time steps. We propose Diagnet, a novel RNN design which incorporates architectural and process-oriented elements which, in combination, greatly increase the stability and quality of the error gradient. We demonstrate in a variety of benchmarking tasks that learning becomes possible with delays of over 10,000 time steps, representing substantial gains over existing RNN designs.  Diagnet’s architecture relies upon three core features: 1) A single layer of segregated recurrent units which are self-connected 2) The absolute value function (an unbounded norm-preserving idempotent nonlinearity), and 3) Application of constraints in both recurrent parameter weights and gradient norms, which tightly control the exploding gradient problem.  Despite the segregated and simplistic nature of Diagnet's recurrent units, significant performance improvement is seen in the 1000-example version of the bAbI task over LSTM and GRU. In addition, Diagnet scores higher in testing accuracy on 15 out of bAbI 20  bAbI tasks, by margins of 3-30\%, while supporting more hidden layers than many other RNN architectures. 

%  The abstract paragraph should be indented \nicefrac{1}{2}~inch
%  (3~picas) on both the left- and right-hand margins. Use 10~point
%  type, with a vertical spacing (leading) of 11~points.  The word
%  \textbf{Abstract} must be centered, bold, and in point size 12. Two
%  line spaces precede the abstract. The abstract must be limited to
%  one paragraph.
\end{abstract}

\section{Introduction}

%NIPS requires electronic submissions.  The electronic submission site
%is
%\begin{center}
%  \url{https://cmt.research.microsoft.com/NIPS2018/}
%\end{center}
%
%Please read the instructions below carefully and follow them faithfully.
Importance of RNNs
Recurrent Neural Networks represent one of the most important deep learning architectures in current use.  They are used for many tasks, such as time series prediction [REF] , unsegmented, connected handwriting recognition [REF], large-vocabulary speech recognition [REF], modeling chaotic phenomena [REF], language modeling and machine translation [REF], robot control [REF], Protein Homology Detection [REF], difficulties of training RNNs [REF], and vanishing/exploding gradient problems [REF]. 
Wikipedia: "A major problem with gradient descent for standard RNN architectures is that error gradients vanish exponentially quickly with the size of the time lag between important events.[34][67] LSTM combined with a BPTT/RTRL hybrid learning method attempts to overcome these problems.[6] "
Vanishing gradients due to the nonlinearities used (e.g. ReLU)
      Problem space We are interested in addressing two widely-cited [RED] pain points of RNN performance.  The first being the vanishing and exploding gradient problem [REF], and the second being performance over increasing time steps [REF].  We address these issues by implementing three as kjfhakdf that have not yet been exibited in combination in neural net design. 
What we found Results were...impressive
Paper Outline in brief In the following paper, we present a brief outline of foundational work in the problem space to contextualize Diagnet’s emergence in RNN design.  Then, we present a discussion of the architecture and mechanisms of Diagnet, along with a useful visual tool for building an intuitive understanding of Diagnet’s units and their behavior.  Next we present a series of common benchmarking tasks, the results of each task, and finally conclude with a discussion of Diagnet’s performance and behavior based on experimental results.  We hope to offer an interesting and powerful new tool for sequence modeling in our introduction of Diagnet, while laying the groundwork for further exploration, architectural innovation, and interaction with the ML community.

\subsection{Style}

Papers to be submitted to NIPS 2018 must be prepared according to the
instructions presented here. Papers may only be up to eight pages
long, including figures. Additional pages \emph{containing only
  acknowledgments and/or cited references} are allowed. Papers that
exceed eight pages of content (ignoring references) will not be
reviewed, or in any other way considered for presentation at the
conference.

The margins in 2018 are the same as since 2007, which allow for
$\sim$$15\%$ more words in the paper compared to earlier years.

Authors are required to use the NIPS \LaTeX{} style files obtainable
at the NIPS website as indicated below. Please make sure you use the
current files and not previous versions. Tweaking the style files may
be grounds for rejection.

\subsection{Retrieval of style files}

The style files for NIPS and other conference information are
available on the World Wide Web at
\begin{center}
  \url{http://www.nips.cc/}
\end{center}
The file \verb+nips_2018.pdf+ contains these instructions and
illustrates the various formatting requirements your NIPS paper must
satisfy.

The only supported style file for NIPS 2018 is \verb+nips_2018.sty+,
rewritten for \LaTeXe{}.  \textbf{Previous style files for \LaTeX{}
  2.09, Microsoft Word, and RTF are no longer supported!}

The \LaTeX{} style file contains three optional arguments: \verb+final+,
which creates a camera-ready copy, \verb+preprint+, which creates a
preprint for submission to, e.g., arXiv, and \verb+nonatbib+, which will
not load the \verb+natbib+ package for you in case of package clash.

\paragraph{New preprint option for 2018}
If you wish to post a preprint of your work online, e.g., on arXiv,
using the NIPS style, please use the \verb+preprint+ option. This will
create a nonanonymized version of your work with the text
``Preprint. Work in progress.''  in the footer. This version may be
distributed as you see fit. Please \textbf{do not} use the
\verb+final+ option, which should \textbf{only} be used for papers
accepted to NIPS.

At submission time, please omit the \verb+final+ and \verb+preprint+
options. This will anonymize your submission and add line numbers to aid
review. Please do \emph{not} refer to these line numbers in your paper
as they will be removed during generation of camera-ready copies.

The file \verb+nips_2018.tex+ may be used as a ``shell'' for writing
your paper. All you have to do is replace the author, title, abstract,
and text of the paper with your own.

The formatting instructions contained in these style files are
summarized in Sections \ref{gen_inst}, \ref{headings}, and
\ref{others} below.

\section{Previous Work}
\label{gen_inst}
It is important to consider a new deep learning architecture with an awareness of existing models.  We offer a brief and by no means comprehensive overview of existing RNN designs herein.  Some of the earliest work consists of Hopfield Networks, developed by Hopfield in 1982.  Subsequently, Schmidhuber [REF] developed the Neural History Compressor (1993) and was able to preserve information over time steps, while compressing information.  In 1991 Fallman developed the 

Recurrent Cascade Correlation Algorithm (1991)--Fallman
Add 1 hidden neuron at a time.  All else frozen.  Inputs go here.  Output of prev goes to new and also all inputs. Each new hidden neuron must be maximally correlated with current remaining err per step.  Now make recurrent: each neuron has single link to self.  Rinse n repeat algo; truncated backprop at the time bc feasible.  At time, proved not useful for certain grammars like seq parity (using saturating, not abs val fxn) First paper T knows of where neuron self-connected. (IndRNN first layer is the same)
LSTM (1997)-gradient preservation. first really good thing for v/exp gradient problem.  Forgets/gates. 
GRU (2014)-just ref it, no significant influence
IRNN (2015)-identity RNN--HINTON, recurrent neurons are relus, init at ID matrix. 2 big ideas. Prob: not used bc unstable, need right params.  Good init results > LSTM simpler architecture. Init at ID matrix was the big idea (unitary matrix) Diagnet too--but off-dig values can't ever be anything but 0; diag are all 1 fast Hadamard product instead of tiresome n\^2.
SCRN (2015)--diagonal entries connected constrained to 1 or a min val, so self-recurrent.  Recognizing value of separate way of handling given recurrent unit connections to self.  Better task results but not less complex computationally. Higher quality learning process.
uRNN (2015)-Bengio et al.  addresses vanishing gradient prob, not exploding.  Transition matrix is unitary.  Enorced.  Complex numbers, etc. Decent results.
DizzyRNN (2016)-abs val. weight matrix unitary transformations, broken into pairs of 2, use 2d rotations. reasonable results. 1-2 tasks 
IndRNN  (2018)--uses self-connection in first layer, loses much efficiency.  Many-to-many connection. Very expensive.  Uses relu for non-linearity rather than our absval.  
Gradient clipping--2 uses here--1)take norm of gradient over all params and limit to N; throw away.  If larger, rescale.  2) Gradients on indiv params, scale down if exceeds limit.  Lossy, loses vector direction information. Directionality is better preserved in this multi-targeted approach. First pass then prune outliers.    

Besides architectural innovations, progress has been made in controlling the gradient directly.

Clipped Gradients (Paskanu, Mikolov, Bengio)

Make a chart of architectures vs features?
Make a feature matrix showing what overlaps the various solutions have? That might be of more interest, and faster than simply talking about all of them individually. We can still cite them.

%The text must be confined within a rectangle 5.5~inches (33~picas)
%wide and 9~inches (54~picas) long. The left margin is 1.5~inch
%(9~picas).  Use 10~point type with a vertical spacing (leading) of
%11~points.  Times New Roman is the preferred typeface throughout, and
%will be selected for you by default.  Paragraphs are separated by
%\nicefrac{1}{2}~line space (5.5 points), with no indentation.
%
%The paper title should be 17~point, initial caps/lower case, bold,
%centered between two horizontal rules. The top rule should be 4~points
%thick and the bottom rule should be 1~point thick. Allow
%\nicefrac{1}{4}~inch space above and below the title to rules. All
%pages should start at 1~inch (6~picas) from the top of the page.
%
%For the final version, authors' names are set in boldface, and each
%name is centered above the corresponding address. The lead author's
%name is to be listed first (left-most), and the co-authors' names (if
%different address) are set to follow. If there is only one co-author,
%list both author and co-author side by side.
%
%Please pay special attention to the instructions in Section \ref{others}
%regarding figures, tables, acknowledgments, and references.

\section{Architecture}
\label{headings}
\begin{figure}
  \centering
  \fbox{\rule[-.5cm]{0cm}{4cm} \rule[-.5cm]{4cm}{0cm}}
  \caption{Figure showing nice diagram of the architecture.}
\end{figure}

Densely connected from input to hidden layer.
Only a single, “shallow” hidden layer.
Hidden layer has no biases.
Hidden layer is recurrently connected by a diagonal matrix; thus every hidden neuron is solely recurrently self-connected.
Self connections of hidden neurons are clamped to lie in the range [-1.0, 1.0]
Self connections of hidden neurons are initialized at 1.0.
Each hidden neuron makes use of the absolute value function as a nonlinearity.
There are no gates* (see “poor man’s gate” section later for an architectural variant)
Densely connected from hidden layer to output
%All headings should be lower case (except for first word and proper
%nouns), flush left, and bold.

%First-level headings should be in 12-point type.

\subsection{Absolute Value Function}

The absolute value function was chosen because it is has a number of desirable properties: 
It is nonlinear (a requirement for interesting computations to be possible)
It is unbounded. (Can I make an argument for why this is an important thing? Something somthing Turing-complete?)
It is idempotent! (This aspect of stability may allow for long term, high precision storage of information without specialized architectural design)
It is (almost) everywhere differentiable, with unit-length derivatives of either -1 or 1.
It is norm-preserving (and hence gradient preserving), both for each individual component of the vector, and also for the entire vector.
The absolute value function cannot, by itself, cause gradients to explode, nor vanish.

Generic restatements about why norm preservation is important when training recurrent neural networks.

LSTM and GRU preserve the gradient over time by mostly avoiding the direct application of nonlinearities as much as possible. (ResNet is similar in this sense.) However, we speculate that many potentially useful algorithms or behaviors likely require the application of a very large number of computations/transformations, rather than simply preserving a small number of computations/transformations over large spans of time.

One toy problem that exhibits such a computational requirement is sequential parity, which will be visited below (and which LSTM and GRU are completely unable to solve for T > 10.)

Others (although surprisingly few) have experimented with using the absolute value function in RNNs, see [REF], [REF]... 
We speculate that it has not received much attention as of yet, because very special care must be taken to control the spectral radius of the recurrent matrix.

Something about absolute value function being the only continuous norm preserving non-linearity (is that true?).
Something about nice properties of absolute value function. Idempotent, etc..

\subsection{Constraints}

Postulate that absolute value function hasn’t been explored much, particularly for recurrent neural nets, because without the proper constraints, its values tend to rapidly blow up.


\subsubsection{Limited [-1, 1] range}
Important to have these constraints, otherwise blows up.
This is an extremely simple way of limiting the spectral radius of the matrix.
In fact (double check) because of the constrained structure, these “recurrent factors” can straightforwardly tell us the singular values, eigenvalues, spectral radius of the linear transformation. (I think the fact that we use no biases helps because it remains linear rather than affine.)

Mention that on some tasks, it often ends up using very different “signatures” of recurrent factor values.

Mention importance of initializing to 1.0 here? Different section?
\subsubsection{Two complementary types of gradient clipping}

Some form of gradient clipping is standard practice for RNNs [REF], but to our knowledge, this is the first application of both global and localized clipping (and it is important to do both!)

Appears to be critical to limit the size of the overall norm of the gradient, particularly for some of the really long-delay tasks.

Critical to do a combination of both global and local norm clipping. Either alone is insufficient.
\subsection{Architectural variations}
\subsubsection{Input masking (“Poor man’s gate”)}

\[h_t = | u \circ h_{t-1} + W\sigma (Vx_t)) |\]

Because the recurrent layer uses no gating mechanism, it can sometimes be useful to have an extra feed forward hidden layer that can learn to selectively ignore certain inputs. In the pathological sequence problems, we have chosen to use ReLU units, because of their hard stop at zero.
This is necessary for all of the “pathological” problems.
“Poor man’s gate” can only mask information that is immediately identifiable as not useful; makes use of no prior contextual information.

%\paragraph{Paragraphs}
%
%There is also a \verb+\paragraph+ command available, which sets the
%heading in bold, flush left, and inline with the text, with the
%heading followed by 1\,em of space.
%
\section{Experiments}
\label{others}
Experimental settings, like learning rates, cooling schedules, hidden layer sizes, etc..

All experiments use the following settings:

RMSprop*, with a learning rate of 0.001
Recurrent parameters are all initialized to 1.0
Feed forward parameters use Xavier’s initialization [REF]
Global gradient norm clipped at 30.0
Per-parameter gradients clipped at [-1.0, 1.0]

Initialize all recurrent factors to 1.0
Derivative clipping
Weight initialization scheme
*Try with mini-batches
Hidden layer sizes
Poor-man’s gate sizes

Maybe make a table of these things for all the experiments? That would be pleasant to read


\subsection{Pathological Sequence Problems}

Maybe mention Hessian-Free learning here, even though it isn’t really the same in terms of architecture.
Reference the original LSTM papers that were the source of the pathological sequence problems.
Mention again the use of the extra feed forward layer before the recurrent one for these.

Reiterate that most of these toy problems emphasize a very small number of computations (applications of nonlinearity) spaced out over very long stretches of time.


%The \verb+natbib+ package will be loaded for you by default.
%Citations may be author/year or numeric, as long as you maintain
%internal consistency.  As to the format of the references themselves,
%any style is acceptable as long as it is used consistently.

%The documentation for \verb+natbib+ may be found at
%\begin{center}
%  \url{http://mirrors.ctan.org/macros/latex/contrib/natbib/natnotes.pdf}
%\end{center}
%Of note is the command \verb+\citet+, which produces citations
%appropriate for use in inline text.  For example,
%\begin{verbatim}
%   \citet{hasselmo} investigated\dots
%\end{verbatim}
%produces
%\begin{quote}
%  Hasselmo, et al.\ (1995) investigated\dots
%\end{quote}
%
%If you wish to load the \verb+natbib+ package with options, you may
%add the following before loading the \verb+nips_2018+ package:
%\begin{verbatim}
%   \PassOptionsToPackage{options}{natbib}
%\end{verbatim}
%
%If \verb+natbib+ clashes with another package you load, you can add
%the optional argument \verb+nonatbib+ when loading the style file:
%\begin{verbatim}
%   \usepackage[nonatbib]{nips_2018}
%\end{verbatim}
%
%As submission is double blind, refer to your own published work in the
%third person. That is, use ``In the previous work of Jones et
%al.\ [4],'' not ``In our previous work [4].'' If you cite your other
%papers that are not widely available (e.g., a journal paper under
%review), use  author names in the citation, e.g., an author
%of the form ``A.\ Anonymous.''

\subsection{Footnotes}

Footnotes should be used sparingly.  If you do require a footnote,
indicate footnotes with a number\footnote{Sample of the first
  footnote.} in the text. Place the footnotes at the bottom of the
page on which they appear.  Precede the footnote with a horizontal
rule of 2~inches (12~picas).

Note that footnotes are properly typeset \emph{after} punctuation
marks.\footnote{As in this example.}

\subsection{Figures}

\begin{figure}
  \centering
  \fbox{\rule[-.5cm]{0cm}{4cm} \rule[-.5cm]{4cm}{0cm}}
  \caption{Figure showing nice diagram of the architecture.}
\end{figure}

All artwork must be neat, clean, and legible. Lines should be dark
enough for purposes of reproduction. The figure number and caption
always appear after the figure. Place one line space before the figure
caption and one line space after the figure. The figure caption should
be lower case (except for first word and proper nouns); figures are
numbered consecutively.

You may use color figures.  However, it is best for the figure
captions and the paper body to be legible if the paper is printed in
either black/white or in color.

\subsection{Tables}

All tables must be centered, neat, clean and legible.  The table
number and title always appear before the table.  See
Table~\ref{sample-table}.

Place one line space before the table title, one line space after the
table title, and one line space after the table. The table title must
be lower case (except for first word and proper nouns); tables are
numbered consecutively.

Note that publication-quality tables \emph{do not contain vertical
  rules.} We strongly suggest the use of the \verb+booktabs+ package,
which allows for typesetting high-quality, professional tables:
\begin{center}
  \url{https://www.ctan.org/pkg/booktabs}
\end{center}
This package was used to typeset Table~\ref{sample-table}.

\begin{table}
  \caption{Sample table title}
  \label{sample-table}
  \centering
  \begin{tabular}{lll}
    \toprule
    \multicolumn{2}{c}{Part}                   \\
    \cmidrule(r){1-2}
    Name     & Description     & Size ($\mu$m) \\
    \midrule
    Dendrite & Input terminal  & $\sim$100     \\
    Axon     & Output terminal & $\sim$10      \\
    Soma     & Cell body       & up to $10^6$  \\
    \bottomrule
  \end{tabular}
\end{table}

\section{Final instructions}

Do not change any aspects of the formatting parameters in the style
files.  In particular, do not modify the width or length of the
rectangle the text should fit into, and do not change font sizes
(except perhaps in the \textbf{References} section; see below). Please
note that pages should be numbered.

\section{Preparing PDF files}

Please prepare submission files with paper size ``US Letter,'' and
not, for example, ``A4.''

Fonts were the main cause of problems in the past years. Your PDF file
must only contain Type 1 or Embedded TrueType fonts. Here are a few
instructions to achieve this.

\begin{itemize}

\item You should directly generate PDF files using \verb+pdflatex+.

\item You can check which fonts a PDF files uses.  In Acrobat Reader,
  select the menu Files$>$Document Properties$>$Fonts and select Show
  All Fonts. You can also use the program \verb+pdffonts+ which comes
  with \verb+xpdf+ and is available out-of-the-box on most Linux
  machines.

\item The IEEE has recommendations for generating PDF files whose
  fonts are also acceptable for NIPS. Please see
  \url{http://www.emfield.org/icuwb2010/downloads/IEEE-PDF-SpecV32.pdf}

\item \verb+xfig+ "patterned" shapes are implemented with bitmap
  fonts.  Use "solid" shapes instead.

\item The \verb+\bbold+ package almost always uses bitmap fonts.  You
  should use the equivalent AMS Fonts:
\begin{verbatim}
   \usepackage{amsfonts}
\end{verbatim}
followed by, e.g., \verb+\mathbb{R}+, \verb+\mathbb{N}+, or
\verb+\mathbb{C}+ for $\mathbb{R}$, $\mathbb{N}$ or $\mathbb{C}$.  You
can also use the following workaround for reals, natural and complex:
\begin{verbatim}
   \newcommand{\RR}{I\!\!R} %real numbers
   \newcommand{\Nat}{I\!\!N} %natural numbers
   \newcommand{\CC}{I\!\!\!\!C} %complex numbers
\end{verbatim}
Note that \verb+amsfonts+ is automatically loaded by the
\verb+amssymb+ package.

\end{itemize}

If your file contains type 3 fonts or non embedded TrueType fonts, we
will ask you to fix it.

\subsection{Margins in \LaTeX{}}

Most of the margin problems come from figures positioned by hand using
\verb+\special+ or other commands. We suggest using the command
\verb+\includegraphics+ from the \verb+graphicx+ package. Always
specify the figure width as a multiple of the line width as in the
example below:
\begin{verbatim}
   \usepackage[pdftex]{graphicx} ...
   \includegraphics[width=0.8\linewidth]{myfile.pdf}
\end{verbatim}
See Section 4.4 in the graphics bundle documentation
(\url{http://mirrors.ctan.org/macros/latex/required/graphics/grfguide.pdf})

A number of width problems arise when \LaTeX{} cannot properly
hyphenate a line. Please give LaTeX hyphenation hints using the
\verb+\-+ command when necessary.

\subsubsection*{Acknowledgments}

Use unnumbered third level headings for the acknowledgments. All
acknowledgments go at the end of the paper. Do not include
acknowledgments in the anonymized submission, only in the final paper.

\section*{References}

References follow the acknowledgments. Use unnumbered first-level
heading for the references. Any choice of citation style is acceptable
as long as you are consistent. It is permissible to reduce the font
size to \verb+small+ (9 point) when listing the references. {\bf
  Remember that you can use more than eight pages as long as the
  additional pages contain \emph{only} cited references.}
\medskip

\small

[1] Alexander, J.A.\ \& Mozer, M.C.\ (1995) Template-based algorithms
for connectionist rule extraction. In G.\ Tesauro, D.S.\ Touretzky and
T.K.\ Leen (eds.), {\it Advances in Neural Information Processing
  Systems 7}, pp.\ 609--616. Cambridge, MA: MIT Press.

[2] Bower, J.M.\ \& Beeman, D.\ (1995) {\it The Book of GENESIS:
  Exploring Realistic Neural Models with the GEneral NEural SImulation
  System.}  New York: TELOS/Springer--Verlag.

[3] Hasselmo, M.E., Schnell, E.\ \& Barkai, E.\ (1995) Dynamics of
learning and recall at excitatory recurrent synapses and cholinergic
modulation in rat hippocampal region CA3. {\it Journal of
  Neuroscience} {\bf 15}(7):5249-5262.

\nolinenumbers
\end{document}
\nolinenumbers
