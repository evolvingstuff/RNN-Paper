%\documentclass[fleqn]{article} %aligns equations
\documentclass{article}
\renewcommand{\theequation}{\arabic{equation}}

% ready for submission
\usepackage[preprint]{nips_2018}

% to compile a preprint version, e.g., for submission to arXiv, add
% add the [preprint] option:
% \usepackage[preprint]{nips_2018}

% to compile a camera-ready version, add the [final] option, e.g.:
% \usepackage[final]{nips_2018}

% to avoid loading the natbib package, add option nonatbib:
% \usepackage[nonatbib]{nips_2018}

\usepackage[utf8]{inputenc} % allow utf-8 input
\usepackage[T1]{fontenc}    % use 8-bit T1 fonts
\usepackage{hyperref}       % hyperlinks
\usepackage{url}            % simple URL typesetting
\usepackage{booktabs}       % professional-quality tables
\usepackage{amsfonts}       % blackboard math symbols
\usepackage{nicefrac}       % compact symbols for 1/2, etc.
\usepackage{microtype}      % microtypography
\usepackage{tikz}
\usepackage{standalone}
\usepackage{amsmath}

%\setcitestyle{numbers}

\title{Diagnet: A Fast, Scalable Recurrent Neural Network}
 
\author{
Thomas Lahore\\
\texttt{tom.lahore@gmail.com}\\
\And
Morgan Weaver\\
\texttt{morganjweaver@gmail.com}
}

\begin{document}
\maketitle
\begin{abstract}

%TODO: cite Rumelhart?
Recurrent neural networks are a challenge to train, in large part due to the well known problem of vanishing and exploding gradients. They also incur a greater computational burden than their feedforward counterparts. Attempts to circumvent these problems have generally led in the direction of increasingly complex architectures and training methodologies. We propose Diagnet, a simple RNN model which consists of three core components: 1) an efficient diagonal hidden-to-hidden recurrent matrix, 2) an absolute value activation function, and 3) the application of constraints to the recurrent parameter weights and gradient norms during training. Despite the decoupled and simplistic nature of Diagnet's hidden units, it significantly outperforms Long Short-Term Memory on the 1k version of the bAbI text understanding tasks from Facebook Research. Diagnet acheives higher test set results than LSTM on 16 out of the 20 tasks, by margins of up to 36\% accuracy.

%TODO: mention efficient RTRL implementation in abstract?
  
% and allows for the use of significantly more hidden neurons than standard fully-connected RNNs. 

\end{abstract}

\section{Introduction}

%TODO: probably should cite more recent examples, because the word "increasingly" implies recent stuff. No older than 2016
%TODO: not sure we should refer to all RNNs as being "deep learning", as there are "deep" RNNs, but Diagnot is not.
%TODO: cite automatic differentiation && black box optimization
%TODO: look through things I've cited on Twitter
Recurrent neural networks (RNNs) are a powerful class of machine learning model specifically designed to handle sequential data. RNNs adpatively learn to process sequences one step at a time, passing forward their own prior computations, which are then incorporated with the input at the next step. This behavior produces a deep computational graph which can be optimized using automatic differentiation or other methods. RNNs are increasingly being used across a wide variety of domains, such as time series prediction \citep{Giles2001}, connected handwriting recognition \citep{graves2009offline}, large-vocabulary speech recognition \citep{SakSB14}, modeling of chaotic phenomena \citep{PhysRevLett.120.024102}, machine translation \citep{bahdanau2014neural, mikolov2010recurrent}, robot control \citep{oubbati2005kinematic}, and protein homology detection \citep{el2008predicting}.

We are interested in addressing two widely-cited pain points of RNN performance.  The first being the vanishing and exploding gradient problem \citep{pascanu2013difficulty, bengio2013advances, pascanu2012understanding}, and the second being performance over increasing time steps \citep{Gers2000, hoch97}.  We address these issues by implementing three elements that have not yet been exhibited in combination in neural net design. 

%TODO: second point above should be about computational efficiency, not training accuracy

Paragraph about vanishing gradients. Mention that most architectures (gated) try to avoid the total number of applications of nonlinearity. Notable exception to that being IRNN and uRNN.

Paragraph about computational burden.

%In the following paper, we present a brief outline of foundational work in the problem space to historically %contextualize Diagnet’s architecture.  Then, we present a discussion of the architecture and mechanisms of Diagnet, %along with a visual tool for building an intuitive understanding of Diagnet’s units and their behavior.  Next we %present a series of common benchmarking tasks, the results of each task, and finally conclude with a discussion of %Diagnet’s performance and behavior based on experimental results.  We hope to offer an interesting and powerful new %tool for sequence modeling in our introduction of Diagnet, while laying the groundwork for exploration in different %problem spaces, architectural innovation, and interaction with the Machine Learning community.

%\paragraph{New preprint option for 2018}
%If you wish to post a preprint of your work online, e.g., on arXiv,
%using the NIPS style, please use the \verb+preprint+ option. This will
%create a nonanonymized version of your work with the text
%``Preprint. Work in progress.''  in the footer. This version may be
%distributed as you see fit. Please \textbf{do not} use the
%\verb+final+ option, which should \textbf{only} be used for papers
%accepted to NIPS.
%
%At submission time, please omit the \verb+final+ and \verb+preprint+
%options. This will anonymize your submission and add line numbers to aid
%review. Please do \emph{not} refer to these line numbers in your paper
%as they will be removed during generation of camera-ready copies.
%
%The file \verb+nips_2018.tex+ may be used as a ``shell'' for writing
%your paper. All you have to do is replace the author, title, abstract,
%and text of the paper with your own.
%
%The formatting instructions contained in these style files are
%summarized in Sections \ref{gen_inst}, \ref{headings}, and
%\ref{others} below.

%TODO: a "we present" paragraph


\section{Architecture}
\label{headings}

%TODO: link to github repository somewhere...

%TODO: give a brief, few sentence long description of the architecture here...

Densely connected from input to hidden layer.
Only a single, “shallow” hidden layer.
Hidden layer has no biases.
Hidden layer is recurrently connected by a diagonal matrix; thus every hidden neuron is solely recurrently self-connected.
Self connections of hidden neurons are clamped to lie in the range $\left[ -1.0, 1.0 \right]$
Self connections of hidden neurons are initialized at 1.0.
Each hidden neuron makes use of an absolute value activation function.
There are no gates
Densely connected from hidden layer to output
%All headings should be lower case (except for first word and proper
%nouns), flush left, and bold.

%TODO: mention no bias in hidden layer

%TODO: make analogy of LSTM, but with all the gates and recurrent weights stripped away, leaving just the CEC

%TODO: Where did the arrows go?!?

\begin{figure}[h]
  \centering
  \includestandalone[scale=0.8]{figures/diagnet1}
  \caption{Diagnet structural diagram.}
\end{figure}

\subsection{Diagonal Recurrent Matrix}

Talk about the computational advantages, as well as suggesting why this might produce a good prior for generalization.

Hadamard Product.

Sparseness appears to help with generalization on NLP tasks, as shown in \citet{2018arXiv180808720D}, where the use of block diagonal matrices greatly reduces the parameter count, while actually boosting test set performance.

Also mention connection to leaky integrators.

Maybe mention something about possibility for much more efficient version of RTRL.

%TODO: wisdom of crowds?

%\subsection{No Recurrent Bias}

%Mention that while not strictly necessary, it appears to help performance.

Mention single hidden layer

Number of operations scales linearly with hidden layer size.


\subsection{Absolute Value Activation Function}

The absolute value activation function has a number of desirable properties: 

%\begin{equation}
%|x| =
%\left\{
%	\begin{array}{ll}
%		\hfill x,  & \mbox{if } x \geq 0 \\
%		-x, & \mbox{if } x < 0
%	\end{array}
%\right.\label{eq:1}
%\end{equation}

%\begin{equation}
%\frac{d|x|}{dx} =
%\frac{x}{|x|} =
%\textrm{sgn}(x) =
%\left\{
%	\begin{array}{ll}
%		\hfill 1,  & \mbox{if } x > 0 \\
%		-1, & \mbox{if } x < 0
%	\end{array}
%\right.\label{eq:1}
%\end{equation}


%\begin{equation}
%\textrm{abs}(\mathbf{x}) = \left \langle |x_1|, |x_2|, \dots , |x_k| \right \rangle
%\end{equation}

%TODO: make this a footnote?
%Note that abs() represents the vectorized version of the absolute value function.

\begin{itemize}
\item \textbf{Idempotent}: In the absence of new input, and with self-recurrent connections of 1.0, long term, high precision storage of information is possible without any specialized architectural design.

%\begin{equation}
%\textrm{abs}(\mathbf{x}) =
%\textrm{abs}(\textrm{abs}(\mathbf{x}))
%\end{equation}

\item \textbf{Forward pass norm-preserving}: Application of the absolute value activation function preserves the length of its vector argument. This is true both element-wise \eqref{eq:1}, and for the vector as a whole \eqref{eq:2}.
%\begin{equation}
%\sqrt{{x_i}^{2}} = 
%\sqrt{{ | x_i | }^{2}} = 
%| x_i |\label{eq:1}
%\end{equation}

\begin{equation}
\sqrt{{x_i}^{2}} = 
| x_i |\label{eq:1}
\end{equation}

\begin{equation}
{\left \| \mathbf{x} \right \|}_2 = 
\sqrt{{x_1}^{2} + \dots + {x_k}^2} = 
\sqrt{{|x_1|}^{2} + \dots + {|x_k|}^2} =
{\left \| \textrm{abs}( \mathbf{x} ) \right \|}_2\label{eq:2}
\end{equation}
\item \textbf{Backward pass norm-preserving}: Similarly, corrective gradient vectors travelling backward through the unfolded computational graph enjoy the same norm-preserving benefits as do the activation vectors in the forward direction, as shown in \eqref{eq:3}.

%\begin{equation}
%\sqrt{{ \left [ \frac{\partial f}{\partial x_i} \right ] }^{2}} = 
%\sqrt{{ \left [ \textrm{sgn} \left ( x_i \right ) \frac{\partial f}{\partial x_i} \right ] }^{2}}
%\end{equation}

\begin{equation}
\sqrt{{ \left [ \frac{\partial f}{\partial x_1} \right ] }^{2} + \dots + { \left [ \frac{\partial f}{\partial x_k} \right ] }^{2}} = 
\sqrt{{\left [ \textrm{sgn} \left ( x_1 \right ) \frac{\partial f}{\partial x_1} \right ] }^{2} + \dots + {\left [ \textrm{sgn} \left ( x_k \right ) \frac{\partial f}{\partial x_k} \right ] }^{2}}\label{eq:3}
\end{equation}

\end{itemize}

These norm-preserving properties ensure that the absolute value activation function cannot be a source of exploding or vanishing gradients.

Mention lack of bias vector to preserve idempotency.

%TODO: Generic restatements about why norm preservation is important when training recurrent neural networks.

\subsection{Constrained Recurrent Weights}

One significant danger in the use of the absolute value activation function is that, without special care and caution, the magnitude of the hidden state can violently explode. If the recurrent matrix has a spectral radius greater than 1, a cycle of unbounded self-amplification can quickly lead to numeric overflow. Fortunately this problem is avoidable by constraining the magnitude of recurrent weights to the range $\left [-1, 1 \right ]$.\footnote{Somewhat unexpectedly, some tasks can be solved even if the recurrent weights are held constant at 1, though it is more often the case that allowing the weights to change with learning leads to better models.
}



%In fact (double check) because of the constrained structure, these “recurrent factors” can straightforwardly tell us the singular values, eigenvalues, spectral radius of the linear transformation. (I think the fact that we use no biases helps because it remains linear rather than affine.)


\subsection{Two Complementary Forms of Gradient Clipping}

Some form of gradient clipping is standard practice for RNNs [REF], and Diagnet uses both global and local clipping, which appears to be critical to limit the size of the overall norm of the gradient, particularly for longer-delay tasks. We have found that either global or local clipping alone is insufficient.

\subsection{Efficient Real-Time Recurrent Learning}

%TODO: is it inefficient in number of operations, or just in memory size of params * hidden nodes?
Real-time recurrent learning (RTRL) is a online approach to training RNNs \citep{RTRL}. Instead of building a deep computational graph extending back into the past, all partial derivatives are updated and carried forward with each timestep. It is not currently in use due to computational inefficiency as compared to backpropagation through time (BPTT). Modern RNNs are generally trained using BPTT. 

However, Diagnet's decoupled hidden layer eliminates the exchange of any information between hidden neurons, and a fortunate side-effect is that there exists a significantly more memory efficient RTRL implementation.

\begin{equation}
{\left[ {\frac{\partial f_j}{\partial w_j}} \right]}_t := \textrm{sgn}\left( \left[ \sum_{i, j} x_{i_t} w_{ij }\right] + r_j f_{j_{t-1}} \right) x_{j_t} + r_j {\left[ {\frac{\partial f_j}{\partial w_j}} \right]}_{t-1}\label{eq:4}
\end{equation}

\begin{equation}
{\left[ {\frac{\partial f_j}{\partial r_j}} \right]}_t := \textrm{sgn}\left( \left[ \sum_{i, j} x_{i_t} w_{ij }\right] + r_j f_{j_{t-1}} \right) f_{j_{t-1}} + r_j {\left[ {\frac{\partial f_j}{\partial r_j}} \right]}_{t-1}\label{eq:5}
\end{equation}

RTRL for fully connected RNNs requires the storage of $n^3 + mn^2$ partial derivatives, whereas for Diagnet that requirement is reduced to $n^2 + mn$.


%TODO: cite this https://dl.acm.org/citation.cfm?id=1351135

\section{Experimental Results}
\label{others}
\subsection{Experimental Configuration} Diagnet is initialized as follows.  We use RMSProp [REF] as a gradient descent optimizer, with a learning rate of .001.  All recurrent parameters are initialized to 1.0, and feedforward parameters use Xavier's initialization [REF]. The global gradient norm is clipped at 30.0, while per-parameter gradients are clipped at [-1,1].  Finally, an unbiased hidden layer allows for the idempotency of the absolute value function. 
%\begin{table}[ht]
%\caption{Hooray for a random table} % title of Table
%\centering % used for centering table
%\begin{tabular}{c c c c} % centered columns (4 columns)
%\hline\hline %inserts double horizontal lines
%Case & Method\#1 & Method\#2 & Method\#3 & Method\#4 \\ [0.5ex] % inserts table
%%heading
%\hline % inserts single horizontal line
%1 & 50 & 837 & 970 \\ % inserting body of the table
%2 & 47 & 877 & 230 \\
%3 & 31 & 25 & 415 \\
%4 & 35 & 144 & 2356 \\
%5 & 45 & 300 & 556 \\ [1ex] % [1ex] adds vertical space
%\hline %inserts single line
%\end{tabular}
%\label{table:nonlin} % is used to refer this table in the text
%\end{table}

bAbI/footnote{https://research.fb.com/downloads/babi/} is a set of 20 synthetic text understanding tasks created by \citet{2015arXiv150205698W} at Facebook Research. Each task is designed to probe language-learning models for particular logical and reasoning capabilities, such as induction, deduction, chaining of facts, and temporal ordering of events. It is suggested that solving these "toy" problems is a necessary (though not sufficient) precondition to the development of more broadly intelligent systems. 

\begin{table}[h]
\caption{Sample from Single Supporting Fact task}
\centering
\begin{tabular}{|l|l|l|}
\hline
Fact & Question & Answer \\
\hline
Mary moved to the bathroom. & & \\
John went to the hallway. & & \\
 & Where is Mary? & bathroom \\
Daniel went back to the hallway. & & \\
Sandra moved to the garden. & & \\
 & Where is Daniel? & hallway \\
John moved to the office. & & \\
Sandra journeyed to the bathroom. & & \\
& Where is Daniel? & hallway \\
Mary moved to the hallway. & & \\
Daniel travelled to the office. & & \\
& Where is Daniel? & office \\
John went back to the garden. & & \\
John moved to the bedroom. & & \\
& Where is Sandra? & bathroom \\
\hline
\end{tabular}
\end{table}

Weston and coleagues showed that a basic LSTM network was unable to solve any of the 20 tasks, where success had been defined as attaining an accuracy of 95\% or more.

The first notable improvement as compared to a vanilla RNN was in the form of MemNNs, previously introduced in [REF]. Since then a number of improvements, such as [REF][REF][REF] have emerged, falling into two overall domains: augmented memories, and attentional mechanisms.

%Augmented memories are a family of approaches that involve fitting a neural network (generally a recurrent one) with an "external" - and differently structured - form of memory. These external memory stores are generally more inspired by computer science than biology, and include concepts such as content addressable memory, random access memory, stacks, queues, Turing tapes, cellular automata, etc.. Almost all external memories make use of fully differentiable read and write mechanisms, although there are a few notable exceptions that are discrete in nature; requiring reinforcement learning rather than pure backpropagation in order to train.

%The other significant family of approaches involve attentional mechanisms. Attentional mechanisms have met with great recent success, particularly in the domain of machine translation. Rather than having the neural network improve its ability to store information initially, attentional mechanisms all the model to go back over the sequential history of its inputs, often multiple times, conditioned on the type of output that is expected. In the case of machine translation, that conditioning will be driven by the next word in the sentence it is translating combined with what it has already produced. In the case of natural language understanding, it is the question being asked that conditions the attention to the relevant subset of facts in the preceding material. In essence, attention allows models to selectively reread do further processing on what has already been seen.
\subsection{Results}

Here we present Diagnet's results on the 20 babI question answering tasks.  Note that two sets of LSTM results are provided: the first, as reported in Weston et al. against bAbI v1.0, and the second, being a standard LSTM tested against bAbI v1.2.  The latter is a reflection of the correction of errors found in the original task set [REF].

\begin{table}[h]

\centering
\caption{bAbI test set accuracies}
\begin{tabular}{|l|r|r|r|r|r|}
\hline
Task & LSTM 1 & LSTM 2 & ReLU-Diagnet & Abs-Diagnet \\
\hline
Single Supporting Fact & 50 & 52.1  & 52.0 &  \textbf{72.1} \\
Two Supporting Facts & 20 & \textbf{37.0}  & 33.6 & 31.8 \\
Three Supporting Facts & 20 & 20.5 & 29.3 & \textbf{32.4} \\
Two Arg. Relations & 61 & 62.9 & 66.9 & \textbf{98.9} \\
Three Arg. Relations & 70 & 61.9 & 70.9 & \textbf{82.4}\\
Yes/No Questions & 48 & 50.7 & 71.1 & \textbf{72.4} \\
Counting & 49 & 78.9 & 75.0 & \textbf{82.1} \\
Lists/Sets & 45 & \textbf{77.2} & 69.1 & 75.1 \\
Simple Negation & 64 & 64.0 & 68.6 &\textbf{77.0} \\
Indefinite Knowledge & 44 & 47.7 & 60.3 & \textbf{66.8} \\
Basic Coreference & 72 & 74.9  & 72.2 & \textbf{81.3} \\
Conjunction & 74 & 76.4 & 72.4 & \textbf{90.4} \\
Compound Coreference & 94 & \textbf{94.4}  & 93.8 & 94.2 \\
Time Reasoning & 27 & 34.8 & \textbf{41.3} & 39.6  \\
Basic Deduction & 21 & 32.4 & 48.8 & \textbf{49.1} \\
Basic Induction & 23 & \textbf{50.6} & 48.6 & 46.5 \\
Positional Reasoning &  51 & 49.1 & \textbf{51.7} & 50.1  \\
Size Reasoning & 52 & 90.8 & 89.6 & \textbf{92.0}  \\
Path Finding & 8 & 9.0 & 8.1 & \textbf{12.4} \\
Agent's Motivations & 91 & 90.7 & \textbf{96.7} & 96.3  \\
\hline

\end{tabular}

\end{table}

As far as we are aware, these are the strongest results to date using a vanilla neural architecture on the weakly supervised version (with 1000 training items per task) of the bAbI tasks.

Note that we are not comparing Diagnet directly to any of the aforementioned “augmented” neural architectures. Instead, we demonstrate that Diagnet significantly outperforms LSTM on many of these 20 tasks, despite using a much simpler and computationally efficient internal architecture.
%Mention that on some tasks, it often ends up using very different “signatures” of recurrent factor values.

\section{Related Work}
\label{gen_inst}

%Break this out by features

%%%%%%%%%%%%%%%%%%%%%%%%
%Diagonal hidden state
%%%%%%%%%%%%%%%%%%%%%%%%

%\textbf{Diagonal hidden state}

%It is important to consider a new deep learning architecture with an awareness of existing models.  We offer a brief and by no means comprehensive overview of existing RNN designs herein.  
%Some of the earliest work consists of Hopfield Networks, developed by Hopfield in 1982.  Subsequently, Schmidhuber \citet{6795261} developed the Neural History Compressor and was able to preserve information over time steps, while compressing information.  

In 1991 Fallman developed the Recurrent Cascade Correlation Algorithm \citet{Fahlman1990TheRC}, in which a single, previously trained hidden unit is added to the network at a time, with incoming weights frozen as the incoming unit is connected in a feed-forward manner with the previous input and hidden units. Each new hidden neuron must be maximally correlated with current remaining error per step.  
Which used saturated functions as opposed to the non-saturating absolute value function.  At that point in time, the design was not feasible for certain tasks like sequential parity.  proved not useful for certain grammars like seq parity (using saturating, not absolute value function). This is the first occurrence we are aware of in the research in which a model contains self-connected neurons.
%Truncated backprop at the time bc feasible. 

IRNN (2015)-identity RNN--HINTON, recurrent neurons are relus, init at ID matrix. 2 big ideas. Prob: not used bc unstable, need right params.  Good init results > LSTM simpler architecture. Init at ID matrix was the big idea (unitary matrix) Diagnet too--but off-dig values can't ever be anything but 0; diag are all 1 fast Hadamard product instead of tiresome n\^2. 

SCRN (2015)--At the same time, Structurally Constrained Recurrent Networks (SCRN) were also introduced to the research community \citet{MikolovJCMR14}. diagonal entries connected constrained to 1 or a min val, so self-recurrent.  Recognizing value of separate way of handling given recurrent unit connections to self.  Better task results but not less complex computationally. Higher quality learning process.

2015 saw the introduction of the Independent RNN \citet{LeJH15}, in which the model's (where nodes consist of rectified linear units) recurrent weight matrix is initialized with the identity matrix.  One caveat of this model is that the parameters need careful tuning for the model to remain stable.  The IndRNN showed better results than LSTM on some tasks, but with a simpler architecture.  In Diagnet, however, off-diagonal values in the identity matrix can only ever be zero, and are one-fast Hadamard products, reducing time complexity significantly. 
%IndRNN  (2018)--uses self-connection in first layer, loses much efficiency.  Many-to-many connection. Very expensive.  Uses relu for nonlinearity rather than our absval.  

%%%%%%%%%%%%%%%%%%%%%%%%%%
%Absolute value function
%%%%%%%%%%%%%%%%%%%%%%%%%%

%\textbf{Absolute value activation function}

Commonly used activation functions like sigmoid, hyperbolic tangent, and ReLU are not norm-preserving. Gated RNN architectures like LSTM and GRU implicitly adopt a strategy of preserving gradients over time by minimizing the degree to which hidden state transformations - both linear and nonlinear - are applied. ResNet \citep{2015arXiv151203385H} and Highway Networks \citep{2015arXiv150500387S} make use of this same design principle. However, stands to reason that many potentially useful algorithms and dynamical behaviors require the application of a long chain of complex transformations, rather than simply preserving a small handful of them over extensive spans of time.

%TODO: mention this or not?
One toy problem that requires a long, unbroken chain of hidden state transformations is sequential parity, which will be visited below (and which LSTM and GRU are completely unable to solve for T > 10.)
 
%Something about absolute value function being the only continuous norm preserving nonlinearity (is that true?).

Others \citep{DorobantuSR16} [REF] have experimented with using the absolute value function in neural networks, but historically this has been a largely unexplored activation function.

DizzyRNN (2016)-abs val. weight matrix unitary transformations, broken into pairs of 2, use 2d rotations. reasonable results. 1-2 tasks

We speculate that the absolute value function has not received much attention as of yet, mostly because special care must be taken to control the spectral radius of the recurrent matrix.


%%%%%%%%%%%%%%
%Constraints 
%%%%%%%%%%%%%%

%\textbf{Constraints and gradient clipping}

%Need gradient clipping stuff here

SELF-CONNECTION: (IndRNN IRNN, SCRN, RCCA) 
The first RNN to incorporate self-connected neurons appeared in 1991, where Fallman developed the Recurrent Cascade Correlation Algorithm \citet{Fahlman1990TheRC}  In this paradigm, one hidden unit is added to the network at a time, with incoming weights frozen as the new unit is connected in a feed-forward manner with the previous output units. Each new hidden neuron must be maximally correlated with current remaining error per step.  This process was repeated until convergence or an acceptable level of error was reached.  At this point in time, this approach was not feasible for certain tasks like sequential parity.  This is the first occurrence we are aware of in the research in which a model contains self-connected neurons.  

DIAGONAL MATRIX: IndRNN, IRNN, RCCA

In 2015, the Independent RNN \citet{DBLP:journals/corr/LeJH15} was proposed, which shares strategies with Diagnet in the inclusion of a diagonal matrix, though not an identity matrix. The IndRNN's nodes consist of rectified linear units, unlike Diagnet, which uses the absolute value function.
%%%IRNN (2015)-identity RNN--HINTON, recurrent neurons are relus not absval, init at ID matrix, then changes, . 2 big ideas. Prob: not used bc unstable, need right params.  Good init results > LSTM simpler architecture. Init at ID matrix was the big idea (unitary matrix) Diagnet too--but off-dig values can't ever be anything but 0; diag are all 1 fast Hadamard product instead of tiresome n\^2. 

ABSVAL: DizzyRNN, ONE TOM IS RSRCHING[???]

At roughly the same time, Structurally Constrained Recurrent Networks (SCRN) were introduced \citet{DBLP:journals/corr/MikolovJCMR14}. Here, diagonal entries in the weight matrix are constrained to 1, with self-recurrent connections

uRNN (2015)-Bengio et al.  addresses vanishing gradient prob, not exploding.  Transition matrix is unitary.  Enforced.  Complex numbers, etc. Decent results. Norm-preserving but not overlapping int prop.

DizzyRNN [REF] is another architecture that utilizes the absolute value function in 
DizzyRNN (2016)-abs val. weight matrix unitary transformations, broken into pairs of 2, use 2d rotations. reasonable results. 1-2 tasks 

Gradient clipping--2 uses here--1)take norm of gradient over all params and limit to N; throw away.  If larger, rescale.  2) Gradients on indiv params, scale down if exceeds limit.  Lossy, loses vector direction information. Directionality is better preserved in this multi-targeted approach. First pass then prune outliers.   


\section{Discussion}
\subsection{Future Work}
Diagnet's architecture suggests many avenues for additional tasks and performance-enhancing modifications.  Its segregated unit design is ideal for parallelization, and we have yet to tap its potential for GPU operations.  Another intuitive addition would be to include an attentional mechanism or memory augmentation, which we leave to future publications.  
As RNNs have historically been utilized with great success in NLP tasks, we intend to continue exploring Diagnet's capabilities in this problem space.  Another area in which complex long-range sequences are of great interest is computational genomics. The roughly 1.5-gigabytes of information in the human genome may benefit from an RNN architecture supporting wider hidden layers.  
Another avenue for exploration would consist of adding layers to the model, for higher-level feature detection.  We suspect that in its current iteration, Diagnet is better suited to some formal grammars than others.  Comparing performance on context-sensitive vs. context-free grammars would be one example of this. 

\section{Conclusion}
Here we outline the features, mechanisms, and benchmarking performance of Diagnet. Results were robust over a variety of tasks, most notably in bAbI question-answering tasks, where Diagnet’s accuracy scores showed up to 40\% percent improvement over Long Short-Term Memory.  While memory augmentation and attentional mechanisms offer clear advantages as architectural additions, we choose to focus on the core RNN functionality in this paper.  Future work may incorporate attentional mechanisms, memory augmentation, and additional task spaces such as semantic embeddings, genetic sequencing, and computer vision.  Diagnet’s novel architecture offers great flexibility in hidden layering, and number of steps in various memory tasks, as well as potential for distributed computing.  Due to the fact that Diagnet does not have gating mechanisms in its current iteration, it is largely input-driven, and tends to perform best in tasks where an output sequence is expected after arbitrary delays. The work presented here represents a first glance at Diagnet’s mechanisms and potential, which the authors hope to expand in future work. 

%
%\subsection{Footnotes}
%
%Footnotes should be used sparingly.  If you do require a footnote,
%indicate footnotes with a number\footnote{Sample of the first
%  footnote.} in the text. Place the footnotes at the bottom of the
%page on which they appear.  Precede the footnote with a horizontal
%rule of 2~inches (12~picas).
%
%Note that footnotes are properly typeset \emph{after} punctuation
%marks.\footnote{As in this example.}
%
%\subsection{Figures}
%
%\begin{figure}
%  \centering
%  \fbox{\rule[-.5cm]{0cm}{4cm} \rule[-.5cm]{4cm}{0cm}}
%  \caption{Figure showing nice diagram of the architecture.}
%\end{figure}
%
%All artwork must be neat, clean, and legible. Lines should be dark
%enough for purposes of reproduction. The figure number and caption
%always appear after the figure. Place one line space before the figure
%caption and one line space after the figure. The figure caption should
%be lower case (except for first word and proper nouns); figures are
%numbered consecutively.
%
%You may use color figures.  However, it is best for the figure
%captions and the paper body to be legible if the paper is printed in
%either black/white or in color.
%
%\subsection{Tables}
%
%All tables must be centered, neat, clean and legible.  The table
%number and title always appear before the table.  See
%Table~\ref{sample-table}.
%
%Place one line space before the table title, one line space after the
%table title, and one line space after the table. The table title must
%be lower case (except for first word and proper nouns); tables are
%numbered consecutively.
%
%Note that publication-quality tables \emph{do not contain vertical
%  rules.} We strongly suggest the use of the \verb+booktabs+ package,
%which allows for typesetting high-quality, professional tables:
%\begin{center}
%  \url{https://www.ctan.org/pkg/booktabs}
%\end{center}
%This package was used to typeset Table~\ref{sample-table}.
%
%\begin{table}
%  \caption{Sample table title}
%  \label{sample-table}
%  \centering
%  \begin{tabular}{lll}
%    \toprule
%    \multicolumn{2}{c}{Part}                   \\
%    \cmidrule(r){1-2}
%    Name     & Description     & Size ($\mu$m) \\
%    \midrule
%    Dendrite & Input terminal  & $\sim$100     \\
%    Axon     & Output terminal & $\sim$10      \\
%    Soma     & Cell body       & up to $10^6$  \\
%    \bottomrule
%  \end{tabular}
%\end{table}
%
%\section{Final instructions}
%
%Do not change any aspects of the formatting parameters in the style
%files.  In particular, do not modify the width or length of the
%rectangle the text should fit into, and do not change font sizes
%(except perhaps in the \textbf{References} section; see below). Please
%note that pages should be numbered.
%
%\section{Preparing PDF files}
%
%Please prepare submission files with paper size ``US Letter,'' and
%not, for example, ``A4.''
%
%Fonts were the main cause of problems in the past years. Your PDF file
%must only contain Type 1 or Embedded TrueType fonts. Here are a few
%instructions to achieve this.
%
%\begin{itemize}
%
%\item You should directly generate PDF files using \verb+pdflatex+.
%
%\item You can check which fonts a PDF files uses.  In Acrobat Reader,
%  select the menu Files$>$Document Properties$>$Fonts and select Show
%  All Fonts. You can also use the program \verb+pdffonts+ which comes
%  with \verb+xpdf+ and is available out-of-the-box on most Linux
%  machines.
%
%\item The IEEE has recommendations for generating PDF files whose
%  fonts are also acceptable for NIPS. Please see
%  \url{http://www.emfield.org/icuwb2010/downloads/IEEE-PDF-SpecV32.pdf}
%
%\item \verb+xfig+ "patterned" shapes are implemented with bitmap
%  fonts.  Use "solid" shapes instead.
%
%\item The \verb+\bbold+ package almost always uses bitmap fonts.  You
%  should use the equivalent AMS Fonts:
%\begin{verbatim}
%   \usepackage{amsfonts}
%\end{verbatim}
%followed by, e.g., \verb+\mathbb{R}+, \verb+\mathbb{N}+, or
%\verb+\mathbb{C}+ for $\mathbb{R}$, $\mathbb{N}$ or $\mathbb{C}$.  You
%can also use the following workaround for reals, natural and complex:
%\begin{verbatim}
%   \newcommand{\RR}{I\!\!R} %real numbers
%   \newcommand{\Nat}{I\!\!N} %natural numbers
%   \newcommand{\CC}{I\!\!\!\!C} %complex numbers
%\end{verbatim}
%Note that \verb+amsfonts+ is automatically loaded by the
%\verb+amssymb+ package.
%
%\end{itemize}
%
%If your file contains type 3 fonts or non embedded TrueType fonts, we
%will ask you to fix it.
%
%\subsection{Margins in \LaTeX{}}
%
%Most of the margin problems come from figures positioned by hand using
%\verb+\special+ or other commands. We suggest using the command
%\verb+\includegraphics+ from the \verb+graphicx+ package. Always
%specify the figure width as a multiple of the line width as in the
%example below:
%\begin{verbatim}
%   \usepackage[pdftex]{graphicx} ...
%   \includegraphics[width=0.8\linewidth]{myfile.pdf}
%\end{verbatim}
%See Section 4.4 in the graphics bundle documentation
%(\url{http://mirrors.ctan.org/macros/latex/required/graphics/grfguide.pdf})
%
%A number of width problems arise when \LaTeX{} cannot properly
%hyphenate a line. Please give LaTeX hyphenation hints using the
%\verb+\-+ command when necessary.

%\subsubsection*{Acknowledgments}
%
%The authors would like to express their gratitude to their sponsor.

\bibliographystyle{plainnat}
\bibliography{refs}

\end{document}
